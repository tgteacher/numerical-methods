\documentclass[12pt]{article}

\usepackage[margin=1in]{geometry}
\usepackage{amsmath,amsthm,amssymb}
\usepackage{mathtools}
\usepackage{multicol}
\usepackage{textcomp}
\usepackage{float}
\usepackage{longtable}

\newcommand{\N}{\mathbb{N}}
\newcommand{\Z}{\mathbb{Z}}
\newcommand\aug{\fboxsep=-\fboxrule\!\!\!\fbox{\strut}\!\!\!}

\newenvironment{theorem}[2][Theorem]{\begin{trivlist}
\item[\hskip \labelsep {\bfseries #1}\hskip \labelsep {\bfseries #2.}]}{\end{trivlist}}
\newenvironment{lemma}[2][Lemma]{\begin{trivlist}
\item[\hskip \labelsep {\bfseries #1}\hskip \labelsep {\bfseries #2.}]}{\end{trivlist}}
\newenvironment{exercise}[2][Exercise]{\begin{trivlist}
\item[\hskip \labelsep {\bfseries #1}\hskip \labelsep {\bfseries #2.}]}{\end{trivlist}}
\newenvironment{reflection}[2][Reflection]{\begin{trivlist}
\item[\hskip \labelsep {\bfseries #1}\hskip \labelsep {\bfseries #2.}]}{\end{trivlist}}
\newenvironment{proposition}[2][Proposition]{\begin{trivlist}
\item[\hskip \labelsep {\bfseries #1}\hskip \labelsep {\bfseries #2.}]}{\end{trivlist}}
\newenvironment{corollary}[2][Corollary]{\begin{trivlist}
\item[\hskip \labelsep {\bfseries #1}\hskip \labelsep {\bfseries #2.}]}{\end{trivlist}}

\begin{document}

\title{TUTORIAL 1}%replace X with the appropriate number
\author{TRISTAN GLATARD\\ %replace with your name
COMP 361 Numerial Methods} %if necessary, replace with your course title
\date{September 13, 2019}
\maketitle

\section{Exercises of today}

\begin{exercise}{1} %You can use theorem, proposition, exercise, or reflection here.
By evaluating the determinant, classify the following matrices as singular, ill-conditioned, or well-conditioned:\\
\begin{center}

%A
$\textbf{A}=
\begin{bmatrix}
1&2&3 \\4&5&6 \\ 7&8&9
\end{bmatrix}$
%B
$\textbf{B}=
\begin{bmatrix}
2&-2&1 \\1&0&-1 \\ 4&1&1
\end{bmatrix}$
%C
$\textbf{C}=
\begin{bmatrix}
1&2.0001&3 \\4&5&6 \\ 7&8&9
\end{bmatrix}$
\end{center}
\textbf{Solution}\\

%Calculate det(A)
$
\vert A \vert = 1
\begin{vmatrix}
5&6 \\ 8&9
\end{vmatrix} - 2
\begin{vmatrix}
4&6 \\ 7&9
\end{vmatrix} + 3
\begin{vmatrix}
4&5 \\ 7&8
\end{vmatrix} = 45 - 48 - 2 (36-42) + 3 (32 - 35) = -3 + 2*6 - 3*3 = 0
$

\textit{A is singular.} \\

%Calculate det(B)
$
\vert B \vert = 2
\begin{vmatrix}
0&-1 \\ 1&1
\end{vmatrix} + 2
\begin{vmatrix}
1&-1 \\ 4&1
\end{vmatrix} + 1
\begin{vmatrix}
1&0\\4&1
\end{vmatrix}=2*1+2*5+1=13
$ \\

\textit{B is well-conditioned.} \\

%Calculate det(C)
$
\vert C \vert = 1
\begin{vmatrix}
5&6 \\ 8&9
\end{vmatrix} - 2.0001
\begin{vmatrix}
4&6 \\ 7&9
\end{vmatrix} + 3
\begin{vmatrix}
4&5 \\ 7&8
\end{vmatrix} = -3 + 2.0001*6 - 3*3 = 0.0006 \\
$ \\

As $\vert C \vert << max(c_{ij})$, \textit{C is ill-conditioned.} \\
\end{exercise}

\section{Bonus exercises}

\begin{exercise}{5} %You can use theorem, proposition, exercise, or reflection here.
Compute the condition number of $\textbf{A}$ using the infinity norm:
\begin{center}
\textbf{A}=
$\begin{bmatrix}
2&-2&1\\
1&0&-1\\
4&1&1
\end{bmatrix}
$
\end{center}
\textbf{Solution}\\

cond(A) = $\Vert A \Vert_\infty . \Vert A^{-1} \Vert_\infty = 6 . \Vert A^{-1} \Vert_\infty$ \\
We need to find $A^{-1}$. We can do it by solving equations $AX = I$ by Gauss elimination. But we already have the LU decomposition of \textbf{A} from \textbf{Excercise 2}:
\begin{center}
\textbf{L}=
$\begin{bmatrix}
1&0&0\\
\frac{1}{2}&1&0\\
2&5&1
\end{bmatrix}
$
\textbf{U}=
$\begin{bmatrix}
2&-2&1\\
0&1&-\frac{3}{2}\\
0&0&\frac{13}{2}
\end{bmatrix}
$
\end{center}
Now we need to solve $LUX = I$, where I is the identity matrix:
\begin{center}
$ I = \begin{bmatrix}
1&0&0\\
0&1&0\\
0&0&1
\end{bmatrix}
$
\end{center}

1. Solve $LUX=\begin{bmatrix} 1\\0\\0\end{bmatrix}$\\

1.1. Solve $Ly=\begin{bmatrix} 1\\0\\0\end{bmatrix}$ by forward substitution
\begin{center}
$y_1 = 1; y_2=-\frac{1}{2};y_3=\frac{1}{2}$
\end{center}

1.2. Solve $Ux=y=\begin{bmatrix} 1\\-\frac{1}{2}\\\frac{1}{2}\end{bmatrix}$
by backward substitution
\begin{center}
$x_3=\frac{1}{13};x_2=-\frac{5}{13};x_1=\frac{1}{13}$
\end{center}

2. Solve $LUX=\begin{bmatrix} 0\\1\\0\end{bmatrix}$

2.1 Solve $Ly=\begin{bmatrix} 0\\1\\0\end{bmatrix}$ by forward substitution
\begin{center}
$y_1 = 0; y_2=1;y_3=-5$
\end{center}

2.2 Solve $Ux=y=\begin{bmatrix} 0\\1\\-5\end{bmatrix}$ by backward substitution
\begin{center}
$x_3=-\frac{10}{13};x_2=-\frac{2}{13};x_1=\frac{3}{13}$
\end{center}

3. Solve $LUX=\begin{bmatrix} 0\\0\\1\end{bmatrix}$

3.1. Solve $Ly=\begin{bmatrix} 0\\0\\1\end{bmatrix}$ by forward substitution
\begin{center}
$y_1 = 0; y_2=0;y_3=1$
\end{center}

3.2. Solve $Ux=y=\begin{bmatrix} 0\\0\\1\end{bmatrix}$ by backward substitution
\begin{center}
$x_3=-\frac{2}{13};x_2=-\frac{3}{13};x_1=\frac{3}{13}$
\end{center}

Finally,
\begin{center}
$A^{-1} = \begin{bmatrix}
\frac{1}{13}&\frac{3}{13}&\frac{2}{13}\\
-\frac{5}{13}&-\frac{2}{13}&\frac{3}{13}\\
\frac{1}{13}&-\frac{10}{13}&\frac{2}{13}\\
\end{bmatrix}
$, and $\Vert A^{-1} \Vert _\infty = 1$ \\
\end{center}

\textbf{cond(A) = 6}
\end{exercise}
%EXERCISE 6-----------------------------------------------------
\begin{exercise}{6} %You can use theorem, proposition, exercise, or reflection here.
Invert the following matrix:
\begin{center}
\textbf{A}=
$\begin{bmatrix}
3&1&2\\
1&1&0\\
5&8&9
\end{bmatrix}
$
\end{center}
\textbf{Solution}\\

We will solve $AX=I$ by Gauss elimination\\
\begin{center}
\textbf{[$A \vert I$]}=
$\begin{bmatrix}
3&1&2 &\aug&1&0&0\\
1&1&0 &\aug& 0&1&0\\
5&8&9 &\aug& 0&0&1
\end{bmatrix}
$ \\
\end{center}
$(2) \leftarrow (2) - \frac{1}{3}(1)$\\
$(3) \leftarrow (3) - \frac{5}{3}(1)$\\
\begin{center}
$\begin{bmatrix}
3&1&2 &\aug&1&0&0\\
0&\frac{2}{3}&-\frac{2}{3} &\aug&-\frac{1}{3}&1&0\\
0&\frac{19}{3}&\frac{17}{3} &\aug& -\frac{5}{3}&0&1
\end{bmatrix}
$ \\
\end{center}
$(3) \leftarrow (3) - \frac{19}{2}(2)$\\
\begin{center}
$\begin{bmatrix}
3&1&2 &\aug&1&0&0\\
0&\frac{2}{3}&-\frac{2}{3} &\aug&-\frac{1}{3}&1&0\\
0&0&12 &\aug& \frac{3}{2}&-\frac{19}{2}&1
\end{bmatrix}
$ \\
\end{center}

Back substitution
\begin{itemize}
\item $12x_{31} = \frac{3}{2} \Rightarrow x_{31} = -\frac{1}{8}$
\item $\frac{2}{3}x_{21} = -\frac{1}{3} + \frac{2.1}{3.8} \Rightarrow x_{21} = -\frac{3}{8}$
\item $x_{11} = \frac{1}{3}(1+\frac{3}{8}-\frac{2}{8}) = \frac{3}{8}$
\end{itemize}

\begin{itemize}
\item $12x_{32} = -\frac{19}{2} \Rightarrow x_{32} = -\frac{19}{24}$
\item $x_{22} = -\frac{3}{2}(1-\frac{2}{3}.\frac{19}{24}) = -\frac{17}{24}$
\item $x_{12} = \frac{1}{3}(0-\frac{17}{24}+2\frac{19}{24}) = \frac{7}{24}$
\end{itemize}

\begin{itemize}
\item $12x_{33} = 1 \Rightarrow x_{33} = \frac{1}{12}$
\item $x_{23} = -\frac{3}{2}(0+\frac{2}{3}.\frac{1}{12}) = \frac{1}{12}$
\item $x_{13} = \frac{1}{3}(0-\frac{1}{12}-\frac{2}{12}) = -\frac{1}{12}$
\end{itemize}

Finally,
\begin{center}
$A^{-1} = \begin{bmatrix}
\frac{3}{8}&\frac{7}{24}&-\frac{1}{12}\\
-\frac{3}{8}&-\frac{17}{24}&\frac{1}{12}\\
\frac{1}{8}&-\frac{19}{24}&\frac{1}{12}\\
\end{bmatrix}
$
\end{center}
\end{exercise}

\end{document}
