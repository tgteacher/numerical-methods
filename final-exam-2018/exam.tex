\documentclass{llncs}

\usepackage{amsmath} % for equation*
\usepackage{wasysym} % for \Box
\usepackage{tipa} % for |
\usepackage{color}
\usepackage{hyperref}
\usepackage{graphicx}
\usepackage{minted}
\definecolor{darkgreen}{rgb}{0,0.7,0}

% Fix link colors
\hypersetup{
    colorlinks = true,
    linkcolor=red,
    citecolor=red,
    urlcolor=blue,
    linktocpage % so that page numbers are clickable in toc
}

\pagestyle{plain}

\newcounter{ques}
\setcounter{ques}{1}

\newcommand{\todo}[1]{\color{blue}\textbf{TODO:} #1\color{black}}
\newcommand{\myspace}[0]{\vspace*{0.25cm}}

\renewcommand{\question}[1]{\paragraph{}\textbf{Q\theques} - #1\stepcounter{ques} }

\newcommand{\answer}[1]{\color{red}\textbf{Solution:}\\#1\color{black}}
\title{COMP 361/5611, Elementary Numerical Methods \\ Final Exam \\ Fall 2018 \\ Friday, December 14, 2018 \\ Duration: 180 minutes}

%\def\arraystretch{2}

\author{Tristan Glatard\\
  \href{mailto:tristan.glatard@concordia.ca}{tristan.glatard@concordia.ca}\\
  \vspace*{0.3cm}
  }

\institute{Concordia University\\
  Department of Computer Science and Software Engineering}

\begin{document}

\maketitle

\section*{Instructions}
\begin{itemize}
\item All questions will receive equal points.
\item Answer all questions on these sheets in the space provided.
\item No books, notes or extra paper (except draft paper received from the instructor).
\item No cell phones, laptops or any other electronic devices except ENCS calculators.
\item This exam is 14 pages long, including the cover page. It has
  12 questions labeled from \textbf{Q1} to \textbf{Q12}. Check that your copy
  is complete.
%~ \item Grading scheme:
%~ \begin{itemize}
%~ \item Good method, good result: 4pts
%~ \item Good method, wrong result (e.g., calculation error): 3pts
%~ \item Wrong method, good result: 2pts
%~ \item Wrong method, wrong result: 0pts
%~ \end{itemize}
\end{itemize}

\myspace

\myspace

\hrulefill\\

\myspace

Consistent with the university regulations concerning cheating and plagiarism I will not cheat during this examination:

\myspace

\myspace

Student ID: \dotfill

\myspace

\myspace

First Name / Last Name: \dotfill

\myspace

\myspace

Signature: \dotfill

\myspace

\myspace

\hrulefill

\newpage
%%% CHAPTER 2
\question{Solve the system of linear equations defined by $\textbf{Ax}=\textbf{b}$ using Gauss elimination, where:
$$
\textbf{A}=
\begin{bmatrix}
4 & -1 & 3 \\
8 & 2 & 3 \\
-2 & 9/2 & -1/2
\end{bmatrix}
\textbf{b} = \begin{bmatrix}
1 \\
3 \\
1/2
\end{bmatrix}
$$
}
\answer{
$$
\textbf{x}=
\begin{bmatrix}
5/16 \\
1/4 \\
0
\end{bmatrix}
$$
Grading:
\begin{itemize}
\item 1pt for method, partial marks if "left side" is fine but right side isn't modified correctly.
\item 1pt for numerical result, partial marks if some iterations are complete.
\end{itemize}
}

\newpage

\question{Use Cholesky decomposition to find \textbf{L} and \textbf{U} such that: 
$$
\textbf{A}=\textbf{LU} =
\begin{bmatrix}
1 & 1 & 1 \\
1 & 2 & 2 \\
1 & 2 & 3
\end{bmatrix}
$$
}

\answer{
$$
\textbf{L}=
\begin{bmatrix}
1 & 0 & 0 \\
1 & 1 & 0 \\
1 & 1 & 1
\end{bmatrix}
;
\textbf{U}=L^{T}
$$
}

\newpage
\question{Explain how the LU decomposition of a matrix is used to solve linear systems.}

\answer{
\begin{enumerate}
\item Solve Ly = b
\item Solve Ux = y 
\end{enumerate}
}


\newpage

\question{Find the line that best fit the following points in the least-square sense:
\begin{center}
\begin{tabular}{|c|c|c|c|}
\hline
x & -1 &  0 & 1 \\
\hline
y & -1 & 0.1 & 0.9\\
\hline
\end{tabular}
\end{center}
}

\answer{
f(x) = 0.95x

Grading:
\begin{itemize}
\item 1pt for method (least-squares formula), no partial marks.
\item 1pt for numerical result, partial marks if some iterations are complete.
\end{itemize}
}


\newpage
%% CHAPTER 4
\question{\emph{Using the Newton-Raphson method}, find the positive zero of function $f(x)=x^2-23$ assuming an initial estimate
of 5 and using an accuracy of 0.001.}
\answer{
x=4.796
Grading:
\begin{itemize}
\item 1pt for method (Newton-Raphson formula), no partial marks.
\item 1pt for numerical result, partial marks if some iterations are complete.
\end{itemize}
}

\newpage
\question{Write the code of the following Python function that finds the root of function f bracketed in [xmin, xmax] using the incremental search method:}
\begin{minted}{python}
from numpy import sign
def root_find_incremental(f, xmin, xmax, delta_x):
    '''
    f is the function for which we will find a zero
    xmin and xmax define the bracket
    delta_x is the step used in incremental search
    Returns (a+b)/2, where the root is in [a, b] and |b-a| < delta_x
    '''
 
\end{minted}

\answer{
}


\newpage
%% CHAPTER 5
\question{Using the central difference approximation of $f''$ in 
$\mathcal{O}(h^2)$, evaluate the derivative of f(x)=$\sqrt{x}$ at $x=0.5$ with 4 decimal places, 
for $h=0.1$. Evaluate the error of your 
approximation with respect to the exact value of f''(x).}

\answer{
$f''(0.5)\approx -0.7161$

Error: $f''(x) = -1/4x^{-3/2}$ ; $f''(0.5)=-0.7071$
Grading:
\begin{itemize}
\item 1pt for method (central differences formula), no partial marks.
\item 0.5pt for numerical result, no partial marks
\item 0.5pt for error
\end{itemize}
}

\newpage
\question{Using the 2-node Gauss-Legendre method, evaluate the following integral and evaluate the error by comparison to the exact result:
$$
\int_{-1}^1 t^2dt
$$
Note: the Gauss-Legendre table is given in Appendix~\ref{app:gl}.
}
\answer{
I=2/3, result is exact.
Grading:
\begin{itemize}
\item 1pt for method, no partial mark.
\item 0.5pt for numerical result, no partial marks.
\item 0.5pt for error.
\end{itemize}
}

\newpage
\question{Estimate the integral of the previous question using the composite trapezoidal rule with 10 panels (n=10). Estimate the error w.r.t the exact value
computed in the previous question.}

\answer{The composite trapezoidal rule is given by:
$$
I = \frac{h}{2}\left[f(x_0) + 2f(x_1) + 2f(x_2) + \ldots + 2f(x_{n-1}) + f(x_n)\right]
$$
Here n=10, $x_0$=-1 and $x_n$=$x_{10}$=1. So $h=0.2$ and $x_i$=$0.1i$.
It gives:\\
....
0.2/2*(1+2*0.8**2+2*0.6**2+2*0.4**2+2*0.2**2+2*0**2+2*0.2**2+2*0.4**2+2*0.6**2+2*0.8**2+1)
0.6800000000000002

Grading:
\begin{itemize}
\item 1pt for method, partial marks if formula "looks good".
\item 1pt for numerical result, no partial marks.
\end{itemize}
}


\newpage
\question{Consider the following initial-value problem:
$$
y=\frac{xy'}{2}-1 \quad ; \quad y(1)=0
$$
\begin{enumerate}
\item Verify that $y(x) = x^2-1$ is solution of the problem.
\item Express the problem in the form $y'=f(y, x)$ so that it can be solved by a numerical solver.
\end{enumerate}
}

\answer{
y'=2(y+1)/x
Grading:
\begin{itemize}
\item 1pt for sub-question 1, partial marks for equation and initial condition.
\item 1pt for sub-question 2, no partial marks.
\end{itemize}
}
\newpage

\question{Write a Python program to solve the initial-value problem 
(IVP) defined in the previous question using the second-order 
Runge-Kutta method (RK2) seen in class. Your program must:
\begin{itemize}
\item Implement a function to solve an IVP using the RK2 method, starting from the skeleton below.
\item Use function \texttt{runge\_kutta\_2} to solve the problem.
\end{itemize}
}
\begin{minted}{python}
from numpy import array
def runge_kutta_2(F, x0, y0, x, h):
    ''' 
    Return y(x) given the following initial value problem:
    y' = F(x, y)
    y(x0) = y0 # initial conditions
    h is the increment of x used in integration
    x is the maximal abscissa until which y will be estimated
    '''
















\end{minted}


\answer{
\color{red} Grading:
\begin{itemize}
\item 1pt for \texttt{rk2} function: 0.5pt for overall principle, 0.5pt for correct ki values.
\item 1pt for using the \texttt{rk2} function: 0.5pt for function definition, 0.5pt for correct use with initial condition.
\end{itemize}
}
\newpage

\question{We consider the following two-point boundary problem:
$$
y''+y'+y=2 \quad ; \quad y(0)=0  \quad \mathrm{and} \quad y(1)=1
$$
Solve the problem using the \emph{finite difference method} with 3 nodes (m=2). 
}

\answer{The unknowns of the problem are $y_0$, $y_1$ and $y_2$. We know that $y_0=0$ and $y_2=1$ from the boundary conditions. The finite difference approximation
of $y_1'$ and $y_1''$ gives $y1 \approx 0.43$.
Grading:
\begin{itemize}
\item 1pt for problem formulation, partial marks if principle is good but formulas are approximate.
\item 1pt for numerical result, no partial marks.
\end{itemize}
}



\newpage
\appendix

\section{Gauss-Legendre table}
\label{app:gl}
\begin{tabular}{c|c|c|c}
n & $x_i$                                                                                                          & $A_i$                                                & Error                             \\ 
\hline                                                                                                                                                                                                         
1 & $\pm\frac{1}{\sqrt{3}}$                                                                                        & 1                                                    & $\frac{f^{(4)}(\xi)}{135}$        \\
2 & 0 ; $\pm \sqrt{\frac{3}{5}}$                                                                                   & $\frac{8}{9}$ ; $\frac{5}{9}$                        & $\frac{f^{(6)}(\xi)}{15,750}$     \\
3 & \parbox{5cm}{$\pm \sqrt{\frac{3}{7}-\frac{2}{7}\sqrt{\frac{6}{5}}}$ \\  $\pm \sqrt{\frac{3}{7}+\frac{2}{7}\sqrt{\frac{6}{5}}}$} & \parbox{5cm}{$\frac{18+\sqrt{30}}{36}$ \\ $\frac{18-\sqrt{30}}{36}$} & $\frac{f^{(8)}(\xi)}{3,472,875}$  \\
\end{tabular}
\end{document}
